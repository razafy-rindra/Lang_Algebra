\input{../template.tex}

\begin{document}
    \begin{exercise}
        Show that every group of order $\leq$ 5 is abelian.

        \begin{proof}
            It is clear that a group of order $1$ is abelian. Any group of prime order is cyclic, so we only need to check that all groups of order $4$ are abelian.

            Let $G$ be a group of order $4$, and $x\in G$ with $x\neq e$. So we have \[\text{ord}(x) = \begin{cases}
                2\\
                4
            \end{cases}\]

            Indeed since $1\neq \text{ord}(x)\mid 4$.

            If $\text{ord}(x) = 4$, then $\{e.x.x^2.x^3\} \leq G \Rightarrow G = \langle x \rangle$, so it is abelian. 

            If $G$ has no elements of order $4$, then for all $x\in G$ we have $x^2 = e \Rightarrow x = x^{-1}$, so for all $x,y\in G$ we have\begin{align*}
                (xy)(x^{-1}y^{-1}) &= (xy)(xy)\\
                                   &= {(xy)}^2\\
                                   &= e
            \end{align*}

            Therefore $xy = yx$ for all $x,y\in G$. So $G$ is abelian.

            In all cases we have shown that if the order of $G$ $\leq 5$, we have that $G$ is abelian. 
        \end{proof}
    \end{exercise}

    \begin{exercise}
    Show that there are two-isomorphic groups of order $4$, namely the cyclic one, and the product of two cyclic groups of order $2$.

    \begin{proof}
        Let $G$ be a group of order $4$, assume that it is not cyclic. In this case, from last question we know that $x^2 = e$ for all $x\in G$, so $\{e,x\} = \langle x\rangle\leq G$ let $y\in G\setminus\langle x\rangle$. 
        
        So notice that $xy \not\in \{e,x,y\}$ indeed since $x,y\not=e$ and $x\neq y$. So we see by comparing order $G = \{e,x,y,xy\}$.
        
        Defining the homomorphism\begin{align*}
            \varphi \colon G&\rightarrow \Z/2\Z \times \Z/2\Z\\
                           x&\rightarrow (1,0)\\
                           y&\rightarrow (0,1) 
        \end{align*}

        Since $\varphi(xy) = \varphi(x)+\varphi(y) = (1,1)$, by inspection we can see that $\ker \varphi = \{e\}$ and $\text{im } \varphi = \Z/2\Z \times \Z/2\Z$, so:\begin{equation*}
            G\simeq \Z/2\Z \times \Z/2\Z
        \end{equation*}
    \end{proof}
    \end{exercise}

    \begin{exercise}
        Let $G$ be a group. A \textbf{commutator} in $G$ is an element of the form $aba^{-1}b^{-1}$ with $a,b\in G$. Let $G^c$ be the subgroup generated by the commutators. Then $G^c$ is called the \textbf{commutator subgroup}. Show that $G^c$ is normal. Show that any homomorphism of $G$ into an abelian group factors through $G/G^c$.

        \begin{proof}
            Since ${(aba^{-1}b^{-1})}^{-1} = bab^{-1}a^{-1}$, the set of elements containing all finite products of commutators is a group. Since any subgroup containing all commutators contains this subgroup we see that $G^c = \{x_1x_2\cdots x_n \mid n\in \N \text{ and } x_i \text{ are commutators}\}$.

            Now let $g\in G$ and $aba{-1}b^{-1}$ be a commutator we see that:\begin{equation*}
                g{(aba{-1}b^{-1})}g^{-1} = ({gag^{-1}})({gbg^{-1}})({ga^{-1}g^{-1}})({gb^{-1}g^{-1}}) = zwz^{-1}w^{-1} 
            \end{equation*}
            Where $z = gag^{-1}$ and $w = gbg^{-1}$.

            So we see that for any $g\in G$ and $a\in G^c$, we have:\begin{align*}
                gag^{-1} &= g(x_1x_2\cdots x_n)g^{-1} \text{ for commutators }x_i\\
                         &= (gx_1g^{-1})(gx_2g^{-1})\cdots (g{x_n}g^{-1})\\
                         &\in G^c \text{ since by above observation }gx_ig^{-1} \text{ is a commutator for all }x_i
            \end{align*}

            So $G^c\trianglelefteq G$.

            Now let $A$ be an abelian group and $\varphi\colon G\rightarrow A$ be a homomorphism. First of all we will show that $\varphi$ contains $G^c$ in it's kernel.\begin{align*}
                \varphi(aba^{-1}b^{-1}) &= \varphi(a)\varphi(b)\varphi(a)^{-1}\varphi(b)^{-1}\\
                                        &= e \text{ by commutating elements}
            \end{align*}

            Therefore we see that for all $x\in G^c$, let $x = x_1\cdots x_n$ where $x_i$ are commutators:\begin{align}
                \varphi(x) = \varphi(x_1)\varphi(x_2)\cdots \varphi(x_n) = e \text{ since each }\varphi(x_i) = e
            \end{align}

            So we indeed see that $G^c\leq \ker \varphi$. So now let $\pi \colon G\rightarrow G/G^c$ be the canonical map and let $\tilde{\varphi}\colon G/G^c \rightarrow A$ be the homomorphism given by:\begin{equation*}
                \tilde{\varphi}(xG^c) =\varphi(x) 
            \end{equation*}

            Note we know that this is a homomorphism since $\varphi$ is a homomorphism.

            Since $G^c \leq \ker\varphi$ if $xG^c = yG^c$ we have $xy^{-1}\in G^c$ so we have $\varphi(xy^{=1}) = e \Rightarrow \varphi(x) = \varphi(y)$ so $\tilde{\varphi}(xG^c) = \tilde{\varphi}(yG^c)$, this homomorphism is indeed well-defined.
        
            So we indeed see that there is a homomorphism $\tilde{\varphi}$ such that $\varphi = \tilde{\varphi}\circ \pi$. So $\varphi$ factors through $G^c$.
        \end{proof}
    \end{exercise}

    \begin{exercise}
        Let $H.K$ be subgroups of a finite group $G$ with $K\subseteq N_H$. Show that:\[|HK| = \frac{|H||K|}{|H\cap K|} \]
    \begin{proof}
        Since $K$ is contained in the normalizer of $H$. Recall by an isomorphism theorem:\[K/(H\cap K) \simeq HK/H \]

        So we have:\begin{equation*}
            \frac{|K|}{|H\cap K|} = \frac{|HK|}{|H|} \Rightarrow |HK| = \frac{|H||K|}{|H\cap K|}
        \end{equation*}
    
    \end{proof}
    \end{exercise}

    \begin{exercise}


        \textbf{Goursat's Lemma.} Let $G$, $G'$ be groups and let $H$ be a subgroup of $G\times G'$ such that the projections $p_1\colon H\rightarrow G$ and $p_2\colon H\rightarrow G'$ are surjective. Let $N$ be the kernel of $p_2$ and $N'$ be the kernel of $p_1$. One can identify $N$ as a normal subgroup of $G$, and $N'$ as a normal subgroup of $G'$. Show that the image of $H$ in $G/N\times G'/N'$ is the graph of an isomorphism \[G/N\simeq G'/N'\]
    
    \begin{proof}
        First of all notice that \begin{equation*}
            \ker p_1 = \{(e,b)\in H\}\simeq N' = \{b\in G' \mid (e,b)\in H\}\text{ and }\ker p_2 = \{(a,e')\in H\} \simeq N = \{a\in Ga \mid (a,e')\in H\}
        \end{equation*} 
        Let \begin{equation*}
            \varphi_1\colon G\rightarrow G/N \text{ and } \varphi_2\colon G'\rightarrow G'/N'
        \end{equation*}
        Be the canonical maps.

        Let $\varphi\colon H\rightarrow G/N\times G'/N'$ be given by \[\varphi((g_1,g_2)) = (\varphi_1(g_1),\varphi_2(g_2)) \]
        This is a homomorphism since $\varphi_1$ and $\varphi_2$ are homomorphisms.

        \
        \begin{lemma} \label{lem1}
            If $(xN,x'N'), (yN,y'N')\in \varphi(H)$ then $xN = yN \iff x'N = y'N$.    
        \end{lemma}
        \begin{proof}
        First assume that $xN = yN$:

        \
        
        We have: $(xy^{-1}N,{x'}{y'}^{-1}N') = (N,{x'}{y'}^{-1}N') \in \varphi(H)$. So let $(a,b)\in H$ such that:\begin{equation*}
            (aN,bN') = \varphi(a,b) = (N,{x'}{y'}^{-1}N')
        \end{equation*}
    
        So we see that $aN = N \Rightarrow a\in N \simeq \ker p_2$. This means that $(a,e')\in H$, so we see that $(e,b) = (a,e')^{-1}(a,b)\in H$, so $b\in N'$. Therefore $N' = bN = {x'}{y'}^{-1}N'$ so $x'N' = y'N'$.

        The other direction is similar.
        \end{proof}

        Now we let \begin{equation*}
            \psi\colon G/N \rightarrow G'/N' \text{ be such that } (aN, \psi(aN)) \in \varphi(H) \text{ for all }aN\in G/N
        \end{equation*}

        We will first show that this function makes sense, note that since the projection from $H$ to $G$ for all $xN$, we see that $(x,y)\in H$ for some $y$. So $\varphi(x,y) = (xN,yN')\in \varphi(H)$ so $xN$ is in the projection off $\varphi(H)$ to $G/N$. So we see that the projection is surjective so: 
        for all $aN\in G/N$ there exists a $bN'\in G'/N'$ such that $(aN,bN')\in \varphi(H)$. Furthermore by lemma \ref{lem1} this $bN'$ is unique. Since this $bN'$ exists and is unique then we can let $\psi(aN) = bN'$ and this function is well-defined.

        Now let $aN,cN\in G/N$ since $(aN, \psi(aN)),(cN,\psi(xN)) \in \varphi(H)$ so:\begin{equation*}
            H \ni (aN, \psi(aN))(cN,\psi(xN)) = (acN, \psi(aN)\psi(cN)) \Rightarrow \psi(aNcN) = \psi(acN) = \psi(aN)\psi(cN)
        \end{equation*} 
        So $\psi$ is indeed a homomorphism. Finallly from lemma \ref{lem1} we see that $\psi(aN) = \psi(bN)$ implies that $(aN,\psi(aN)),(bN,\psi(aN))\in \varphi(H)$ so $aN=bN$. So this function is indeed an isomorphism.
    \end{proof}
    \end{exercise}

    \begin{exercise}
        Prove that the group of inner automorphishms of a group $G$ is normal in $\text{Aut}(G)$.
        \begin{proof}
            For all $g\in G$ we let $\varphi_g$ be the homomorphism such that \[\varphi_g(x) = gxg^{-1}\]

            Recall that an inner automorphishms is an automorphishms of the form $\varphi_g$ for some $g\in G$. Now let: $I = \{\varphi_g \mid g\in G\}$.

            Notice that \[\forall x\in G, \ \varphi_a\circ\varphi_b(x) = abxb^{-1}a^{-1} = (ab)x{(ab)}^{-1} = \varphi_{ab}(x)\Rightarrow \varphi_a\circ\varphi_b = \varphi_{ab}\in I \]

            Likewise \[\forall x\in G, \ \varphi_{a^{-1}}\circ\varphi_a(x) = a^{-1}axa^{-1}a = x\Rightarrow {\varphi_a}^{-1} = \varphi_{a^{-1}}\in I \]
        
        Let $f\in \text{Aut}(G)$, let $\varphi_g\in I$, for all $x\in G$:\begin{align*}
            f\circ\varphi_g\circ f^{-1}(x) &= f(gf^{-1}(x)g^{-1})\\
                                           &= f(g)xf(g^{-1}) \text{ since }f\text{ is a homomorphism}\\
                                           &= f(g)x{f(g)}^{-1}\\
                                           &= \varphi_{f(g)}(x)
        \end{align*}

        Since this is true for all $x$ then we have $f\circ\varphi_g\circ f^{-1} \in I$. Since this is true for all $\varphi_g$ we have $fIf^{-1} \subseteq I$, for all $f\in \text{Aut}(G)$. So $I\trianglelefteq \text{Aut}(G)$.


        \end{proof}
    \end{exercise}

    \begin{exercise}
        Let $G$ be a group such that $\text{Aut}(G)$ is cyclic. Prove that $G$ is abelian.
        \begin{proof}
            Let $N$ be the inner automorphishms group, since it is a subgroup of $\text{Aut}(G)$ it is cylic. Now we define:\begin{align*}
                \varphi &\colon G|rightarrow N\\
                        &\varphi(g)\rightarrow \varphi_g
            \end{align*}
            Where $\varphi_g$ is defined as in exersise $6$. Let $Z(G) = \{z\in G \mid zg = gz \ \forall g\in G\}$, it is clear that $Z(G)\subseteq \ker \varphi$. Furthermore if $g\in \ker\varphi$ we have:\begin{equation*}
                \forall x\in G \ x = \text{id}(x) = \varphi_g(x) = gxg^{-1} \ \therefore gx = xg \Rightarrow g\in Z(G)
            \end{equation*}

            So we see that $\ker\varphi = Z(G)$, so we have \[G/Z(G)\simeq N\]
            Since $N$ is cyclic so is $G/Z(G)$, so let $gZ(G)$ be a generator. Let $x.y\in G$ we have $x = {g^m}z$, $y = {g^n}z'$ for some $n,m\in \Z$ and $z,z'\in Z(G)$. We have:\begin{align*}
                xy &= {g^m}z{g^n}z'\\
                    &= {g^m}{g^n}zz'\\
                    &= {g^n}{g^m}z'z\\
                    &=  {g^n}z'{g^m}z\\
                    &= yx
            \end{align*}
            Since $x,y$ are arbitrary we see that $G$ is indeed abelian.
        \end{proof}
    \end{exercise}

    \begin{exercise}
        Let $G$ be a group and let $H,H'$ be subgroups. By a \textbf{double coset} of $H,H'$ one means a subset of $G$ of the form $HxH'$.\begin{enumerate}[label = (\alph*)]
            \item Show that $G$ is a disjoint union of double cosets.
            \item Let $\{c\}$ be a family of representatives for the double cosets. For each $a\in G$ denote by $[a]H'$ the conjugate $aH'a^{-1}$ of $H'$. For each $c$ we have a decomposition into ordinary cosets\[H = \bigcup_{x_c} x_c(H\cap [c]H')\] where $\{x_c\}$ is a family of elements of $H$, depending on $c$. Show that the elements $\{{x_c}c\}$ form a family of left coset representatives for $H'$ in $G$; that is,\[G = \bigcup_{c}\bigcup_{x_c}{x_c}cH',\] and the union is disjoint.
        \end{enumerate}
    \begin{proof}
        \begin{enumerate}[label = (\alph*)]
            \item First of all assume that $z\in HxH'\cap HyH'$ then let $h_1,h_2\in H$ and $h_1', h_2'\in H'$ such that:\begin{align*}
                h_1xh_1' = z = h_2yh_2'\Rightarrow y = {h_2}^{-1}h_1xh_1'{h_2'}^{-1} \Rightarrow HyH' = H{h_2}^{-1}h_1xh_1'{h_2'}^{-1}H' = HxH'
            \end{align*} 

            For any $y,x\in G$ either $HxH'$ and $HyH'$ are disjoint or they are equal this fact combined with the fact that for all $x\in G$ we have $x\in HxH'$ tells us that we can write $G$ as a disjoint union of double cosets.

            \item  By our assumptions we have the disjoint unions:\begin{align*}
                G &= \bigcup_c HcH'\\
                  &= \bigcup_c\bigcup_{x_c}x_c(H\cap [c]H')cH'
            \end{align*}

            But notice that for $\alpha\in x_c(H\cap [c]H')cH'$ we have:\begin{align*}
                \alpha = {x_c}ch'c^{-1}ch'' = {x_c}ch'h''\in {x_c}cH' \text{ for some }h',h''\in H'  
            \end{align*}

            So we see that $x_c(H\cap [c]H')cH'\subseteq {x_c}cH'$, the other inclusion is clear since $e\in (H\cap [c]H')$. So we have:\begin{equation*}
                G = \bigcup_c\bigcup_{x_c}{x_c}cH' \text{ and this union is disjoint} 
            \end{equation*}
        \end{enumerate}
    \end{proof}
    \end{exercise}

    \begin{exercise}
        \begin{enumerate}[label = (\alph*)]
            \item Let $G$ be a group and $H$ a subroup of finite index. Show that there exists a normal subgroup $N$ of $G$ contained in $H$ and also of finite index.
            \item Let $G$ be a group and let $H_1,H_2$ be subgroups of finite index. Prove that $H_1\cap H_2$ has finite index.
        \end{enumerate}
    \begin{proof}
        \begin{enumerate}[label = (\alph*)]
            \item Assume that $[G\colon H] = n$ and let $\{a_1H,a_2H,\dots, {a_n}H\}$ be the distinct cosets of $H$ in $G$. Let $a\in G$ since we know that ${aa_i}H\in \{a_1H,a_2H,\dots, {a_n}H\}$, so let $\sigma_a\in S_n$ be such that ${aa_i}H = {a_{\sigma_a(i)}}H$ for all $i$. We define\begin{align*}
                \varphi\colon &G\rightarrow S_n\\
                              &a\rightarrow \sigma_a
            \end{align*} 

            Let $x\in \ker \varphi$, this means that ${xa_i}H = a_{e(i)}H = {a_i}H$ for all $i$. In particular we know that for one $i_0$ we have ${a_{i_0}}H = H$, so we have \[H={a_{i_0}}H = x{a_{i_0}}H = xH\]

            This means that $x\in H$. Therefore $\ker\varphi \subseteq H$. Letting $N = \ker\varphi$, we see that this is a normal subgroup conatined in $H$.

            Now finally note that $\text{im }\varphi\leq S_n$, so we see by an isomorphism theorem:\begin{equation*}
                [G\colon N] = |G/N| = |\text{im }\varphi| \leq |S_n| = n!<\infty
            \end{equation*}
            \item First of all, let $x,y\in G$ be such that $xH_1 = yH_1$ and $xH_2 = yH_2$, then we have $y^{-1}x\in H_1$ and $y^{-1}x\in H_2$ so $y^{-1}x\in H_1\cap H_2$ so $x(H_1\cap H_2) = y(H_1\cap H_2)$ 
            
            
            Let $a_1H_1,\ldots, {a_n}H_1$ be the distinct cosets of $H_1$ and $b_1H_2,\ldots, {b_m}H_2$ be the distinct cosets of $H_2$.
            
            Let $a\in G$ and let $i$ and $j$ such that $aH_1 = {a_i}H_1$ and $aH_2 = {b_j}H_2$.
        \end{enumerate}
    \end{proof}
    \end{exercise}

    \begin{exercise}
        Let $G$ be a group and let $H$ be a subgroup of finite index. Prove that there is only a finite number of right cosets of $H$, and that the number of right cosets is equal to the number of left cosets.
    \begin{proof}
        Recall, that since $H$ has finite index, There are only a finite number of left cosets of $H$ in $G$.

        \begin{align*}
            ah = b &\iff hb^{-1} = a^{-1}\\
            \therefore \ b\in aH &\iff  a^{-1}\in Hb^{-1}
        \end{align*}

        From this we see that $aH = bH \iff Ha^{-1} = Hb^{-1}$. 

        Let $H_L = \{aH\mid a\in G\}$ the set of left cosets, and let $H_R = \{Ha \mid a\in G\}$ the set of right cosets.

        We define a set map:\[f\colon H_L\rightarrow H_R\] Given by $f(aH) = Ha^{-1}$. Now first we will show that this function is well-defined, let $aH = bH$ this implies from above that $Ha^{-1} = Hb^{-1}$:\begin{equation*}
            f(aH) = Ha^{-1} = Hb^{-1} = f(bH)
        \end{equation*} 
        So this function is indeed well-definied. Now this is also a bijection we notice that the inverse function is given by the map:\[g\colon H_R\rightarrow H_L \text{ by }g(Ha) = a^{-1}H \]
    
        This function is similarly seen to be well-defined since $aH = bH \iff Ha^{-1} = Hb^{-1}$. 

        So we see that $[G\colon H] = |H_L| = |H_R|$. Since $[G\colon H]<\infty$, there is only a finite number of right cosets and there as a many right as left cosets.
    \end{proof}
    \end{exercise}

    \begin{exercise}
        Let $G$ be a group, and $A$ a normal abelian subgroup. Show that $G/A$ operates on $A$ by conjugation; and in this manner get a homomorphism of $G/A$ into $\text{Aut}(A)$.
    \begin{proof}
        We will first show that this action is well-defined: Assume that $xA = yA$ so let $a\in A$ such that $y = xa$ and let $s\in A$. So we have:\begin{align*}
            xA\cdot s &= xsx^{-1}\\
                     &= xaa^{-1}sx^{-1}\\
                     &= (xa)s(a^{-1}x^{-1}) \text{ since }a,s\in A\text{ and }A \text{ is abelian.}\\
                     &= (xa)s{(xa)}^{-1}\\
                     &= ysy^{-1}\\
                     &= yA\cdot s
        \end{align*}  

        So this function is indeed well-defined. Now we will show that this function is a group action:

        \

        Let $xA, yA\in G/A$ and $s\in A$ we have\begin{equation*}
            xA\cdot (yA\cdot s) = xA\cdot (ysy^{-1}) = xysy^{-1}x^{-1} = (xy)s{(xy)}^{-1} = (xyA)\cdot s
        \end{equation*}
    
        \

        For all $s\in S$\begin{equation*}
            eA\cdot s = ese^{-1} = s
        \end{equation*}
    
        So this indeed a group action.

    Now let \begin{align*}
        \varphi\colon G/A &\rightarrow \text{Aut}(A)
    \end{align*}
    Be such that for all $s\in A$: \begin{align*}
        \varphi(xA)(s) &= xA\cdot s\\
        &= xsx^{-1}
    \end{align*}

    Note that it is clear that $\varphi(xA)$ is an automorphishms, since it is an inner-homomorphism.

    Finallly it is clear that $\varphi$ is a homomorphism, since the map $xA\cdot s = xsx^{-1}$ is a group action, so \[\varphi(xyA)(s) = (xyA)\cdot s = xA\cdot (yA\cdot s) = \varphi(xA)\varphi(yA)(s) \Rightarrow \varphi(xy)=\varphi(x)\varphi(y)\]

    And likewise from above we see that $\varphi(A) = \text{id}$.
    \end{proof}
    \end{exercise}
\end{document}
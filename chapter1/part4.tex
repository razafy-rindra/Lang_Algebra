\paragraph*{Part 4: Explicit determination of groups}
%27, 28, 29, 30, 31, 32, 33, 34, 35, 36, 36, 37, 38, 39, 40, 41
\

% QUestion 24

\begin{exercise}
\begin{proof}
    

    Assume that $p$ is prime and let $G$ be a group of order $p^2$. 

    Since each element in the center of $G$ forms a conjugacy class containing just itslef, if $x_1,\ldots,x_r$ are the conjugacy representatives not in the center   \[
    |G| = |Z(G)| + \sum_i |G\colon G_{x_i}| 

\]

Therefore we know that $p\mid |Z(G)|$. So we have $|Z(G)| = \begin{cases}p^2\\p\end{cases}$ 
Assume that $|Z(G)| = p$,  $|G/Z(G)| = p$, so $G/Z(G)$ is cyclic say $G/Z(G) = \langle gZ(G) \rangle$. Now let $x,y\in G$ then we have $x = g^n a$ and $y = g^m b$, where $n,m\in \Z$ and $a,b\in Z(G)$:
\begin{align*}
xy &= (g^n a)(g^m b)\\
   &= g^n g^m a b \text{ since }a\in Z(G)\\
   &= g^{n+m} ba \text{ since }b\in Z(G)\\
    &= g^m b g^n a\\
    &= yx
\end{align*}
Since $x,y$ were arbitrary elements in $G$, then $G$ is abelian which contradicts the fact that $Z(G)\neq G$. $\therefore |Z(G)| = p^2$, so $Z(G) = G$ and $G$ is abelian. 

\ 


Assume that $G$ is not isomorphic to the cyclic group $\Z/p^2\Z$. This means that $\text{ord}(x) = p$ for all $x\in G\setminus\{e\}$.

Let $x\in G\setminus\{e\}$, we define $H$ to be the subgroup of $G$ generated by $x$ and let $y\in G\setminus H$, and $K$ be the subgroup generated by $y$.
Since $K\cap H = \{e\}$, and since $G$ is abelian then from question 12 we conclude that $|HK| = |H|\cdot |K| = p^2$. Therefore $HK = G$ and  \[
   G =  HK\simeq H\times K \simeq \Z/p\Z\times \Z/p\Z$
.\]
\end{proof}
\end{exercise}

%Question 25
\begin{exercise}
    \begin{enumerate}[label = (\alph*)]
    \item
    \begin{proof}
    By the class equation $p\mid |Z|$, therefore since $G$ is not abelian\[
    |Z| = \begin{cases}p\\p^2\end{cases}
    .\]
Assume that $|Z| = p^2$, then $|G/Z| = p$, which would mean that  $G/Z$ is cyclic so $G$ is abelian which is a contradiction. So  $|Z| = p$. Therefore  $Z\simeq C$. Now notice that  $|G/Z| = p^2$, but it is not cyclic since $G$ is
not abelian, so by last question  $G/Z \simeq C\times C$.
    \end{proof}
\item 
\begin{proof}
    
    Since the index of $H$ is $p$ it is normal. Assume for a contradiction that $Z$ is not contained in $H$. Then notice that $Z\cap H = \{e\}$, since $Z = \langle g \rangle$ for some $g\in G$ and $g\not\in H$.
    Therefore since $gh = hg$ for all  $h\in H$ and $g\in Z$, by question 12 we see that $|ZH| = |Z|\cdot |H| = p^3$. So we have that $ZH = G$.
    
    The last thing we need to recall is that from question $24$, $H$ is an abelian group.


    Let $x,y\in G$ there exists  $g_1,g_2\in Z$ and  $h_1,h_2\in H$ such that $x = g_1h_1$ and $y = g_2h_2$. 
    \begin{align*}
    xy &=  (g_1h_1)(g_2h_2)\\
    &= g_2g_1h_1h_2 \text{ since } g_2\in Z\\
    &= g_2g_1h_2h_1 \text{ since }H \text{ is abelian we have }h_2h_1=h_1h_2\\
    &= (g_2h_2)(g_1h_1) \text{ since }g_1\in Z\\
    &= yx
    \end{align*}
    Which implies that $G$ is abelian, which is a contradiction. Therefore,  $Z\subseteq H$.
\end{proof}
\item 
\begin{proof}
Let $x\in G\setminus Z$ and let $K = \langle x \rangle$. We notice that since $K$ and $Z$ are both cyclic groups of prime order and $x\notin Z$ we have $K\cap Z = \{e\}$

Since $Z\trianglelefteq G$ we see that $H = KZ \leq G$. Furthermore from question 12:

\[|H| = |K||Z|  =  p^2.\]

From the previous question this subgroup is normal and since $|H| = p^2$ and $x^p = e$ for all $x\in H$ by question 24, $H \simeq C\times C$.
\end{proof}
\end{enumerate}
\end{exercise}

% Question 26
\begin{exercise}
    \begin{enumerate}[label = (\alph*)]
    \item \begin{proof}
    
    Let $P$ be a Sylow $p$-subgroup of $G$ and $Q$ be a Sylow $q$-subgroup of $G$. Since $Q$ has index $p$, we know that it is a normal subgroup of $G$.
From 14(a), since $P$ has the same order as $G/Q$ and $p,q$ are distinct primes we know that $G = PQ$.

\

Now we let $P$ acts on $Q$ by conjugation. This gives us a homomorphism  \[
    \varphi \colon P \rightarrow Aut(Q)
.\]
Now since $\ker \varphi \leq P$ and $P$ is a simple group we know by the isomorphism theorem that \[|\varphi(P)| = |P|/|\ker \varphi| = \begin{cases}
p\\
1
\end{cases}\]

But furthermore we know that $|\varphi(P)|\ \mid \ |Aut(Q)| = q-1$, so if $|\varphi(P)| = p$ we would have  $q-1 = 0\pmod{p}$, which contradicts our assumption. So we must have that $\varphi$ is trivial.
%Therefore we see that $xy = yx$ for all $x\in P$ and $y\in Q$.
Since $P\cap Q = \{e\}$, from question 12 we know that $P\times Q \simeq PQ$. 
Finally from proposition $4.3(v)$, we know that  $P\times Q$ is cyclic so $G = PQ \simeq P\times Q$ is also cyclic.
\end{proof}
%Since $P\cap Q = \{e\}$, from question 12 we know that that $P\times Q \simeq PQ$, by comparing orders we then know that $G\simeq P\times Q$. We conclude that $G$ is cyclic from proposition $4.3(v)$.
    \item Since $15 = 3\cdot 5$ and  $5 = 2 \pmod 3$. The result is immidiate from last question.
\end{enumerate}
\end{exercise}


% Question 27
\begin{exercise}
 %   We already know that every group of prime order and every group of order $pq$ where $p,q$ are (not necessarily distinct) primes are solvable. Furthermore 25 tells us that any group of order $p^3$ is solvable.
%{12, 16, 18, 20, 24, 28, 30, 32, 36, 40, 42, 45, 48, 50, 52, 54, 56}


%Using 28 and 29 we can reduce this further:
%{16,24, 32, 36, 40, 48, 54, 56}
\end{exercise}



% Question 28
\begin{exercise}
%Let $n_q$ be the number of Sylow q-subgroups of  $G$, recall that if $H$ is a Sylow q-subgroup then \begin{center}        
%$n_q = $ number of conjugate subgroups of $H$ = $|G\colon N_H|$
%\end{center}     
%\therefore$n_q \mid \frac{|G|}{|H|} = \frac{p^2q}{q} = p^2$, since $H\leq N_H$. Therefore we have $n_q = 1,p$ or $p^2$.

%Furthermore recall that $n_q = 1\pmod{q}$. Assume that $n_q\neq 1$, then we have that 
\end{exercise}

% Question 29
\begin{exercise}
    
\end{exercise}

% Question 30

\begin{exercise}
 \begin{enumerate}[label = (\alph*)]
    \item 
\begin{proof}
Since $40 = 5\cdot 2^3$. We have $n_5 | 8$ and  $n_5 = 1 \pmod{5}$ which forces $n_5 = 1$.
\end{proof}
\item
\begin{proof}
Since $12 = 3\cdot 2^2$. %We have $n_4| 2$ and $n_3 = 1 \pmod{3}$ so $n_3 = \{1,4\}$. Assume that $n_3 = 4$, each Sylow 3-subgroup is of prime order so is a cyclic group.
%Therefore they all intersect trivially. So there are $(3-1)\cdot 4 = 8$ elements of order $3$, since we only have room for  $4$ more elements in this group
%there can only be one Sylow 2-subgroup (of order $4$).
This is true from Exercise 28.
\end{proof}
\end{enumerate}
\end{exercise}

% Question 31

\begin{exercise}
%    The only groups of prime order are cyclic, and we know the only two groups of order $p^2$ up to isomorphism. 
%    So we only need to study the groups of order $6,8$ and $10$ in more detail.
 %   \begin{center}
        
%\begin{tabular}{cc}
 %   Order of group & Groups up to isomorphism.\\
  %  \hline
   % 1 & \text{trivial group}\\
   % 2 & \Z/2\Z\\
   % 3 & \Z/3\Z\\
   % 4 & \Z/4\Z, \ \Z/2\Z\times\Z/2\Z\\
   % 5 & \Z/5\Z\\
   % 6 & \\
   % 7 & \Z/7\Z\\
   % 8 & \\
   % 9 & \Z/9\Z, \ \Z/3\Z\times\Z/3\Z\\
   % 10 &\\
%\end{tabular}

 %   \end{center}
\end{exercise}


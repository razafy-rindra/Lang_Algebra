\documentclass{article}
\usepackage[margin=0.5in]{geometry}
\usepackage[utf8]{inputenc}

\usepackage{amsmath}
\usepackage{amsthm}
\usepackage{amssymb}
\usepackage{enumerate}
\usepackage{chngcntr}
\usepackage{mathtools}
\usepackage{enumitem}
\usepackage{listings}
\usepackage{hyperref}
\usepackage[dvipsnames]{xcolor}
\usepackage[bb=boondox]{mathalfa}
\newcommand{\Z}{\mathbb{Z}}
\newcommand{\C}{\mathbb{C}}
\newcommand{\HH}{\mathbb{H}}
\newcommand{\Q}{\mathbb{Q}}
\newcommand{\R}{\mathbb{R}}
\newcommand{\N}{\mathbb{N}}
\newcommand{\F}{\mathbb{F}}
\newcommand{\qbinom}{\genfrac{[}{]}{0pt}{}}
\DeclarePairedDelimiter\ceil{\lceil}{\rceil}
\DeclarePairedDelimiter\floor{\lfloor}{\rfloor}

\usepackage[dvipsnames]{xcolor}

\newtheorem{theorem}{Theorem}
\newtheorem{corollary}{Corollary}[theorem] 
\newtheorem{lemma}[theorem]{Lemma} 
\newtheorem{proposition}{Proposition}
\newcommand{\greenparagraph}[1]{\textcolor{ForestGreen}{\textbf{#1}}}

\theoremstyle{definition}
\newtheorem{definition}{Definition}[section]
\theoremstyle{remark}
\newtheorem*{remark}{Remark}
\theoremstyle{remark}
\newtheorem*{note}{Note}
\theoremstyle{definition}
\newtheorem{example}{Example}[definition]
\newcounter{exercise}[subsection]
\newenvironment{exercise}{\refstepcounter{exercise}\textbf{Exercise~\theexercise}}{}
\counterwithin*{equation}{section}
\counterwithin*{equation}{subsection}
\lstset{
  basicstyle=\ttfamily,
  mathescape
}
\newcommand{\Gal}{\text{Gal}}
\begin{document}
In this chapter we will study the core of Galois theory, the group of automorphisms of a finite (and sometimes infinite) Galois extension at length.
    \section{Galois extensions}
\begin{definition}
    Let $K$ be a field and let $G$ be a group of automorphisms of $K$. We let \[K^G = \{x\in K\mid x^\sigma = x \text{ for all }\sigma \in G\}\]We call this the \textbf{fixed field} of $G$. 
\end{definition}

\begin{definition}
    An algebraic extension $K$ of a field $k$ is called \textbf{Galois} if it is normal and seperable.

    \

    The group of automorphisms of $K$ over $k$ is called the \textbf{Galois group} of $K$ over $k$, and is denoted $G(K/k), \ G_{K/k}, \ \Gal(K/k)$ or simply $G$.
\end{definition}

This is the main theorem of the Galois theory or finite Galois extensions.
\begin{theorem}\label{main}
    Let $K$ be a finite Galois extension of $k$, with Galois group $G$. There is a bijection between the set of subfields $E$ of $K$, containing $K$ and the set of subgroups $H$ of $G$, given by $E=K^H$. The field $E$ is Galois over $k$ if and only if $H$ is normal in $G$, and if that is the case, then the map $\sigma\rightarrow \sigma|_E$ induces an isomorphism
    of $G/H$ onto the Galois group of $E$ over $k$.
\end{theorem}



\end{document}
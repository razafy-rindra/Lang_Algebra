\input{../template.tex}
\usepackage{thmtools}

\declaretheoremstyle[
  headfont=\color{LimeGreen}\normalfont\bfseries,
  bodyfont=\color{ForestGreen}\normalfont\itshape,
]{colored}

\declaretheorem[
  style=colored,
  name=Bergman,
]{bergman}
\newcommand{\Gal}{\text{Gal}}
\newcommand{\Bergman}[1]{\textcolor{LimeGreen}{\textit{#1}}}
\begin{document}
Legend: Everything in \Bergman{green} is from the Bergman's Companion to Lang's Algebra.

\

In this chapter we will study the core of Galois theory, the group of automorphisms of a finite (and sometimes infinite) Galois extension at length.
    \section{Galois extensions}
\begin{definition}
    Let $K$ be a field and let $G$ be a group of automorphisms of $K$. We let \[K^G = \{x\in K\mid x^\sigma = x \text{ for all }\sigma \in G\}\]We call this the \textbf{fixed field} of $G$. 
\end{definition}

\begin{definition}
    An algebraic extension $K$ of a field $k$ is called \textbf{Galois} if it is normal and seperable.

    \

    The group of automorphisms of $K$ over $k$ is called the \textbf{Galois group} of $K$ over $k$, and is denoted $G(K/k), \ G_{K/k}, \ \Gal(K/k)$ or simply $G$.
\end{definition}

This is the main theorem of the Galois theory or finite Galois extensions.
\begin{theorem}\label{main}
    Let $K$ be a finite Galois extension of $k$, with Galois group $G$. There is a bijection between the set of subfields $E$ of $K$, containing $K$ and the set of subgroups $H$ of $G$, given by $E=K^H$. The field $E$ is Galois over $k$ if and only if $H$ is normal in $G$, and if that is the case, then the map $\sigma\rightarrow \sigma|_E$ induces an isomorphism
    of $G/H$ onto the Galois group of $E$ over $k$.
\end{theorem}

\begin{theorem}
    Let $K$ be a Galois extension of $k$. Let $G$ be its Galois group. Then $k=K^G$. If $F$ is an intermediate field, $k\subseteq F\subseteq K$, then $K$ is Galois over $F$. The map\[F\rightarrow \Gal(K/F)\]from the set of intermediate fields into the set of subgroups of $G$ is injective.

    \begin{proof}
        Let $\alpha\in K^G$. Let $\sigma$ be any embedding of $k(\alpha)$ in $K^a$, inducing the identity on $k$. Extend $\sigma$ to an embedding of $K$ into $K^a$, we also call this extension $\sigma$. Note since $K$ is normal, $\sigma$ is an automorphisms of $K$ over $k$ it is an element of $G$. Since $\alpha\in K^G$, $\sigma$ leaves $\alpha$ fixed. Therefore there is actually only one extension of $\sigma$ to an embedding of $K$ in $K^a$ (the identity). So:\[{[k(\alpha)\colon k]}_s = 1 \] 
        Since $\alpha$ is seperable over $k$, $[k(\alpha)\colon k] = {[k(\alpha)\colon k]}_s = 1$, so $\alpha\in k$. This proves the first assertion.

        \

        Let $F$ be an intermediate field. Then $K$ is normal and seperable over $F$ by previous theorems from chapter five. Hence $K$ is Galois over $F$. If $H = \Gal(K/F)$ then by what we have proved above we conclude that $F = K^H$. Now we will show that the map defined in our statement is injective. Let $F,F'$ be intermediate fields such that $F\rightarrow \Gal{K/F} = H$ and $F'\rightarrow \Gal{K/F'} = H'$.

        Assume that $H=H'$, then:\begin{equation*}
            F = K^H = K^{H'} = F'
        \end{equation*}
    \end{proof}
\end{theorem} 
    \begin{definition}
        We shall call the group $\Gal({K/F})$ of an intermediate field the group \textbf{associated} with $F$. We say that a subgroup $H$ of $G$ \textbf{belongs} to an intermediate field $F$ if $H = \Gal({K/F})$
       \begin{bergman}
        Note this does not mean that $H$ is the Galois group of $F$. For example the Galois group of the whole extension $K$ is $\Gal(K/F)$, $\{1\}$ is the subgroup belonging to $K$, since $\{1\} = \Gal(K/K)$.
    \end{bergman}
    \end{definition}

\begin{corollary}
    Let $K/k$ be Galois with group $G$. Let $F,F'$ be two intermediate fields, and let $H,H'$ be the subgroups of $G$ belonging to $F,F'$ respectively. Then $H\cap H'$ belongs to $FF'$.
    \begin{proof}
        Note every element of $H\cap H'$ leaves $FF'$ fixed (basically from how $FF'$ is constructed), and every element of $G$ which also leaves $FF'$ fixed also leaves $F$ and $F'$ fixed so lies in $H\cap H'$.
    \end{proof}
\end{corollary}

\begin{corollary}
    The fixed field of the smallest subgroup of $G$ containing $H$ and $H'$ is $F\cap F'$.
    \begin{proof}
        Recall that $H.H'$ are the subgroups of $G$ belonging to $F,F'$ respectively.
    \end{proof} 
\end{corollary}

\end{document}
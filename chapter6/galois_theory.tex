\documentclass{article}
\usepackage[margin=0.5in]{geometry}
\usepackage[utf8]{inputenc}

\usepackage{amsmath}
\usepackage{amsthm}
\usepackage{amssymb}
\usepackage{enumerate}
\usepackage{chngcntr}
\usepackage{mathtools}
\usepackage{enumitem}
\usepackage{listings}
\usepackage{hyperref}
\usepackage[dvipsnames]{xcolor}
\usepackage[bb=boondox]{mathalfa}
\newcommand{\Z}{\mathbb{Z}}
\newcommand{\C}{\mathbb{C}}
\newcommand{\HH}{\mathbb{H}}
\newcommand{\Q}{\mathbb{Q}}
\newcommand{\R}{\mathbb{R}}
\newcommand{\N}{\mathbb{N}}
\newcommand{\F}{\mathbb{F}}
\newcommand{\qbinom}{\genfrac{[}{]}{0pt}{}}
\DeclarePairedDelimiter\ceil{\lceil}{\rceil}
\DeclarePairedDelimiter\floor{\lfloor}{\rfloor}

\usepackage[dvipsnames]{xcolor}

\newtheorem{theorem}{Theorem}
\newtheorem{corollary}{Corollary}[theorem] 
\newtheorem{lemma}[theorem]{Lemma} 
\newtheorem{proposition}{Proposition}
\newcommand{\greenparagraph}[1]{\textcolor{ForestGreen}{\textbf{#1}}}

\theoremstyle{definition}
\newtheorem{definition}{Definition}[section]
\theoremstyle{remark}
\newtheorem*{remark}{Remark}
\theoremstyle{remark}
\newtheorem*{note}{Note}
\theoremstyle{definition}
\newtheorem{example}{Example}[definition]
\newcounter{exercise}[subsection]
\newenvironment{exercise}{\refstepcounter{exercise}\textbf{Exercise~\theexercise}}{}
\counterwithin*{equation}{section}
\counterwithin*{equation}{subsection}
\lstset{
  basicstyle=\ttfamily,
  mathescape
}
\usepackage{thmtools}
\usepackage{makeidx}
\makeindex

\declaretheoremstyle[
  headfont=\color{LimeGreen}\normalfont\bfseries,
  bodyfont=\color{ForestGreen}\normalfont\itshape,
]{colored}

\declaretheorem[
  style=colored,
  name=Bergman,
]{bergman}
\newcommand{\Gal}{\text{Gal}}
\newcommand{\Bergman}[1]{\textcolor{LimeGreen}{\textit{#1}}}
\begin{document}
Legend: Everything in \Bergman{green} is from the Bergman's Companion to Lang's Algebra.

\

In this chapter we will study the core of Galois theory, the group of automorphisms of a finite (and sometimes infinite) Galois extension at length.
    \section{Galois extensions}
\begin{definition}\index{fixed field}
    Let $K$ be a field and let $G$ be a group of automorphisms of $K$. We let \[K^G = \{x\in K\mid x^\sigma = x \text{ for all }\sigma \in G\}\]We call this the \textbf{fixed field} of $G$. 
\end{definition}

\begin{definition}\index{Galois group}
    An algebraic extension $K$ of a field $k$ is called \textbf{Galois} if it is normal and seperable.

    \

    The group of automorphisms of $K$ over $k$ is called the \textbf{Galois group} of $K$ over $k$, and is denoted $G(K/k), \ G_{K/k}, \ \Gal(K/k)$ or simply $G$.
\end{definition}

This is the main theorem of the Galois theory or finite Galois extensions.
\begin{theorem}\label{main}
    Let $K$ be a finite Galois extension of $k$, with Galois group $G$. There is a bijection between the set of subfields $E$ of $K$, containing $K$ and the set of subgroups $H$ of $G$, given by $E=K^H$. The field $E$ is Galois over $k$ if and only if $H$ is normal in $G$, and if that is the case, then the map $\sigma\rightarrow \sigma|_E$ induces an isomorphism
    of $G/H$ onto the Galois group of $E$ over $k$.
\end{theorem}

\begin{theorem}
    Let $K$ be a Galois extension of $k$. Let $G$ be its Galois group. Then $k=K^G$. If $F$ is an intermediate field, $k\subseteq F\subseteq K$, then $K$ is Galois over $F$. The map\[F\rightarrow \Gal(K/F)\]from the set of intermediate fields into the set of subgroups of $G$ is injective.

    \begin{proof}
        Let $\alpha\in K^G$. Let $\sigma$ be any embedding of $k(\alpha)$ in $K^a$, inducing the identity on $k$. Extend $\sigma$ to an embedding of $K$ into $K^a$, we also call this extension $\sigma$. Note since $K$ is normal, $\sigma$ is an automorphisms of $K$ over $k$ it is an element of $G$. Since $\alpha\in K^G$, $\sigma$ leaves $\alpha$ fixed. Therefore there is actually only one extension of $\sigma$ to an embedding of $K$ in $K^a$ (the identity). So:\[{[k(\alpha)\colon k]}_s = 1 \] 
        Since $\alpha$ is seperable over $k$, $[k(\alpha)\colon k] = {[k(\alpha)\colon k]}_s = 1$, so $\alpha\in k$. This proves the first assertion.

        \

        Let $F$ be an intermediate field. Then $K$ is normal and seperable over $F$ by previous theorems from chapter five. Hence $K$ is Galois over $F$. If $H = \Gal(K/F)$ then by what we have proved above we conclude that $F = K^H$. Now we will show that the map defined in our statement is injective. Let $F,F'$ be intermediate fields such that $F\rightarrow \Gal{K/F} = H$ and $F'\rightarrow \Gal{K/F'} = H'$.

        Assume that $H=H'$, then:\begin{equation*}
            F = K^H = K^{H'} = F'
        \end{equation*}
    \end{proof}
\end{theorem} 
    \begin{definition}\index{associated}\index{belongs to}
        We shall call the group $\Gal({K/F})$ of an intermediate field the group \textbf{associated} with $F$. We say that a subgroup $H$ of $G$ \textbf{belongs} to an intermediate field $F$ if $H = \Gal({K/F})$
       \begin{bergman}
        Note this does not mean that $H$ is the Galois group of $F$. For example the Galois group of the whole extension $K$ is $\Gal(K/F)$, $\{1\}$ is the subgroup belonging to $K$, since $\{1\} = \Gal(K/K)$.
    \end{bergman}
    \end{definition}

\begin{corollary}
    Let $K/k$ be Galois with group $G$. Let $F,F'$ be two intermediate fields, and let $H,H'$ be the subgroups of $G$ belonging to $F,F'$ respectively. Then $H\cap H'$ belongs to $FF'$.
    \begin{proof}
        Note every element of $H\cap H'$ leaves $FF'$ fixed (basically from how $FF'$ is constructed), and every element of $G$ which also leaves $FF'$ fixed also leaves $F$ and $F'$ fixed so lies in $H\cap H'$.
    \end{proof}
\end{corollary}

\begin{corollary}
    The fixed field of the smallest subgroup of $G$ containing $H$ and $H'$ is $F\cap F'$.
    \begin{proof}
        Let $E$ be the smallest subgroup of $G$ containing $H$ and $H'$. Note this means that $E = \langle H\cup H' \rangle$. 
        
        Let $x\in K^E$. This means that \[\sigma(x) = x \text{ for all }\sigma\in E\]
        Since $H,H'\subseteq E$ we see that $x\in K^H = F$ and $x\in K^{H'} = F'$. So $x\in F\cap F'$. On the other hand, if $x\in F\cap F'$, then for $\sigma\in E$ we have $\sigma = \tau_1\cdots \tau_n$, where $\tau_i\in H\cup H'$.

        So \[\sigma(x) = \tau_1\circ\cdots \circ \tau_{n-1}\circ \tau_n(x) = \tau_1\circ\cdots \circ \tau_{n-1}(x) = \ldots = x\] Since $\tau_i(x) =x$ for all $i$,

        Therefore we indeed see that $F\cap F' = K^E$.
    \end{proof} 
\end{corollary}
\begin{corollary}
    $F\subseteq F'$ if and only if $H'\subseteq H$

    \begin{proof}
        If $F\subseteq F'$ and $\sigma\in H'$ leaves $F'$ fixed, then $\sigma$ leaves $F$ fixed, so $\sigma\in H$. So $H'\subseteq H$.

        \

        Conversely if $H'\subseteq H$, then $F = K^{H}\subseteq K^{H'} = F'$. 
    \end{proof}
\end{corollary}

\begin{corollary}
    Let $E$ be a finite seperable extension of a field $k$. Let $K$ be the smallest normal extension of $k$ containing $E$. Then $K$ is finite Galois over $k$. There is only a finite number of intermediate fields $F$ such that $k\subseteq F\subseteq E$.

    \begin{proof}
        Note $K$ is the compositum of a the finite number of conjugates of $E$, i.e \[
            K = (\sigma_1 E)\cdots (\sigma_n E) \text{ where }\sigma_i\text{ are the distinct embeddings of }E\text{ into }E^a    
        \] Therefore it is normal(by definition), seperable(since $E$ is) and it is finite over $k$.

        \

        The Galois group $K/k$ has only a finite number of subgroups. So there is only a finite number of subfields of $K$ containing $k$, so a finite number of subfields of $E$ containing $k$.
    \end{proof}
\end{corollary}

\begin{lemma}\label{1.7}
    Let $E$ be an algebraic seperable extension of $k$. Assume that there is an integer $n\geq 1$ such that every element $\alpha\in E$ is of degree $\leq n$ over $k$. Then $E$ is finite over $k$ and $[E\colon k]\leq n$.

    \begin{proof}
        Let $\alpha\in E$ be such that $m = [k(\alpha)\colon k]\leq n$ is maximal. Assume that, there exists $\beta\in E\setminus k(\alpha)$, then since $k(\alpha,\beta)$ is seperable and finite over $k$ by the primitive element theorem there is a $\gamma\in k(\alpha,\beta)\subseteq E$ such that:

        \[ [k(\gamma)\colon k] = [k(\alpha,\beta)\colon k] > m \] Which contradicts our assumption that $\alpha$ had maximal degree in $E$. Therefore $E\setminus k(\alpha) = \emptyset\Rightarrow E = k(\alpha)$,

        So it is finite over $k$ and $[E\colon k]\leq n$.
    \end{proof}
\end{lemma}

\begin{theorem}\textbf{Artin}
    Let $K$ be a field and let $G$ be a finite group of automorphisms of $K$, of order $n$. Let $k = K^G$ be the fixed field. Then $K$ is a finite Galois extension of $k$, and its Galois group is $G$. We have $[K\colon k] = n$,

    \begin{proof}
        Let $\alpha\in K$ and let $\sigma_1,\ldots,\sigma_r$ be a maximal set of elements of $G$ such that $\sigma_1\alpha,\ldots,\sigma_r\alpha$ are distinct. If $\tau\in G$ then for all $i$, there is a $\xi\in S_r$ such that\[\tau\sigma_i\alpha = \sigma_{\xi(i)}\alpha\]
    
        Indeed $\tau\sigma_i\alpha\in \{\sigma_1\alpha,\ldots,\sigma_r\alpha\}$, by maximality. And since $\tau$ is injective, $\tau\sigma_i\alpha = \tau\sigma_j\alpha\iff \sigma_i\alpha = \sigma_j\alpha$.
  
        So not only is $\alpha$ the root of a polynomial \[f(X) = \prod_{i=1}^r (X-\sigma_i\alpha) \text{ and }\forall \tau\in G \text{, }f^\tau = f \]

        So the coefficients of $f$ are in $K^G = k$. Furthermore, $f$ is seperable since all the $\sigma_i\alpha$ are distinct. So every element $\alpha\in K$ is the root of a seperable polynomial of degree $\leq n$ with coeffs in $k$. We also see that this polynomial splits into linear factors in $K$, so $K$ is seperable and normal (hence Galois) over $k$.

        By lemma~\ref{1.7} we see that $[K\colon k]\leq n$. But recall from chapter $5$, the Galois group of $K$ over $k$ has order $\leq [K\colon k]$. Since $G\subseteq \Gal(K/k)$, but $n = |G| \leq |\Gal(K/k)|\leq [K\colon k]\leq n$, we see that $G = \Gal(K/k)$, and $[K\colon k] = n$. 
    \end{proof}
 \end{theorem}
\printindex
\end{document}
%! Tex root = ./galois_theory.tex
  \paragraph*{Galois connections}

  \begin{bergman}
    The FTG is an example of a general type of mathematical situation
    \begin{definition}
      Let $S,T$ be sets and $R\subseteq S\times T$ be a binary relation on them. For every subset $X\subseteq S$ we define \[
        X^\ast = \{t\in T\mid (\forall s\in X), \ (s,t)\in R\}\subseteq T  
      \]
      And likewise for every subset $Y\subseteq T$ we define\[
        Y^\ast = \{s\in S\mid (\forall t\in Y), \ (s,t)\in R\}\subseteq S  
      \] 

      These operators constitute what is called the \textbf{Galois connection} between $S$ and $T$.
    \end{definition}
    So note in our case, $S$ is an extension field $K$ of $k$, $T = \Gal{K/k}$ and $R$ is the relation \[\{(x,\sigma)\in K\times \Gal{K/k}\mid \sigma(x) = x \}\]
  
  \begin{Properties}
    \item $X\subseteq X' \Rightarrow X^\ast \supseteq X'^\ast$
    \item The operators $\ast\ast$ from $P(S)\rightarrow P(S)$ and  $P(T)\rightarrow P(T)$, are the \textit{closure operators} on $S$ and $T$, i.e.\begin{itemize}
      \item $X\subseteq X^{\ast\ast}$
      \item $X\subseteq Y\Rightarrow X^{\ast\ast}\subseteq Y^{\ast\ast}$
      \item ${(X^{\ast\ast})}^{\ast\ast} = X^{\ast\ast}$
    \end{itemize}
    \item The operators $\ast$ give a bujective inclusion-reversing correspondence between closed subsets of $S$ and closed subsets of $T$ wrt to these operators.
  \end{Properties}
  \end{bergman}

  \section{Examples and Applications}
  \begin{bergman}
    Let $K$ be a finite field, say $\F_{p^n}$, recall the Frobenius map is an automorphism in this case, furthermore we recall that this map fixes $\F_p$. Furthermore the Frobenius map has finite order, so by Artin
    it generates the group $\Gal{\F_{p^n}/\F_{p}}$. Thus each field $\F_{p^n}$ is a Galois extension of $\F_{p}$ with cyclic Galois group, which must have order $[\F_{p^n}, \F_{p}] = n$.

    We can see that for every $m$ dividing $n$, the sugroup generated by the $m$th power of the Frobenius map corresponds to a subfield $\F_{p^m}\subseteq \F_{p^n}$ 
  \end{bergman}
  
  \begin{definition}\index{Galois group}
    Let $k$ be a field and $f(X)$ a seperable polynomial of degree $\geq 1$ in $k[X]$. Let \[
    f(X) = (X-\alpha_1)\cdots (X-\alpha_n)  
\]

    be its factorization in a splitting field $K$ over $k$. If $G$ is the Galois group of $K$ over $k$, we call $G$ the \textbf{Galois group} of $f$ over $k$. 
    Since the elements of $G$ permute the roots of $f$, we have an injective homomorphism of $G$ into the symmetric group $S_n$. 
  \end{definition}
  \paragraph*{Quadratic extensions}
  \begin{example}
    Let $k$ be a field and $a\in k$. If $a$ is not a square in $k$, then the polynomial $X^2-a$ has no root in $k$ and is therefore irreducible.
  
    Assume $char k\neq 2$. Let $\alpha$ be a root, this polynomial is separable since $-\alpha\neq \alpha$, and so $k(\alpha)$ is the splitting field, is Galois and it's Galois group is cyclic of order $2$.
  
    Conversely, given an extension $K$ of $k$ of degree $2$, there exists $a\in k$ such that $K = k(\alpha)$ and $\alpha^2 = a$. (Indeed since if $x^2 + bx + c = 0\Rightarrow {(x + \frac{b}{2})}^2 = -(\frac{b^2}{4} + c)\in k$, so we let $\alpha = x + \frac{b}{2}$)
  \end{example}


	\begin{bergman}
		%Let us develop formally the corresponding result for equations of arbitrary degree. 
    Recall that $D$ and $\delta$ are the polynomials 
		\begin{align*}
			\delta(t) &= \prod_{i<j}(t_i-t_j)\\
			D = D(s_1,\ldots,s_n)&= \prod_{i<j}{(t_i-t_j)}^2
		\end{align*}
    \begin{lemma}
  Let $n$ be a positive integer, and let $D\in \Z[X_1,\ldots,X_n]$ be the discriminant polynomial. That is the unique polynomial such that in the polynomial ring $\Z[t_1,\ldots,t_n]$, if one writes $s_i$ for the $ith$ elementary symmetric polynomial in the $t_i$, we have \[
\prod_{i<j} {(t_i-t_j)}^2 = D(s_1,\ldots,s_n)  
\]

  For any field $k$ not of characteristic $2$ and any seperable monic polynomial $f = X^n + a_{n-1}X^{n-1}+\cdots + a_0$, if we denote by $E$ the splitting field of $f$ and by $\alpha_1,\ldots,\alpha_n$ the roots of $f$ in $E$, then TFAE:\begin{enumerate}[label = (\alph*)]
    \item $\Gal{E/k}$, regarded as a group of permutations of $\alpha_1,\ldots,\alpha_n$ lies in the alternating group $A_n$, i.e. acts by even permutations on these roots.
    \item The element $\delta(\alpha) = \prod_{i<j}(\alpha_i-\alpha_j)$ of $E$ lies in $k$.
    \item The element $D(a_{n-1},\ldots,a_0)\in k$ is  asquare in $k$
  \end{enumerate}
  Hence the fixed field of the group of elements of $\Gal{E/k}$ which act by even permutations on the roots of $f$ is $k(\delta(\alpha)) = k({D(a_{n-1},\ldots,a_0)}^{1/2})$.
\end{lemma}
\begin{proof}
  \begin{itemize}
    \item $(a\Rightarrow b)$. \\ Let $\varphi_{(i,i+1)}$ be the automorphism of $\Z[t_1,\ldots,t_n]$ such that $\varphi_{(i,i+1)}(t_k) = t_{(i,i+1)k}$, where $(i,i+1)$ is a permutation. This automorphism takes $\delta(t) = \prod_{i<j}(t_i-t_j)$ to $-\delta(t)$. Since an odd permutation can be characterised as the product of an odd number of permutations of this form, and an even permutation is characterised by an even number of permutations of this form.
    We see that any odd permutation sends $\delta(t)$ to $-\delta(t)$ and any even permutation sends it to itself.

    \

    If an automorphism $\theta$ of $E$ acts by a permutation $\pi_\theta$ on the roots $\alpha_1,\ldots,\alpha_n$ of $f$, then the map $\Z[t_1,\ldots,t_n] \rightarrow E$ carrying $t_i$ to $\alpha_i$ makes a commuting square with the automorphism $\theta$ to the automorphism of the polynomial ring acting by $\pi_\theta$ on the subscripts of the indeterminates. 

    So $\theta$ will send $\delta(\alpha)$ to itself if $\pi_\theta$ is even and it's opposite if it is odd. Hence if all the members of $\Gal{E/K}$ act by even permutations on $\alpha_1,\ldots,\alpha_n$ then $\delta(\alpha)$ is fixed under that group. Hence belongs to $k$. 
    \item $(b\Rightarrow a)$ Conversely, if $\delta(\alpha)$ belongs to $k$, then all memebers of the Galois group fix it, hence act by even permutations.
    \item $(b\Rightarrow c)$ Since $\delta(\alpha)\in k$ then $D(a_{n-1},\ldots,a_0) = {\delta(\alpha)}^2\in k$.
    \item $(c\Rightarrow b)$ Since $D(a_{n-1},\ldots,a_0)$ is a square in $k$ it has a root in $k$, but $X^2 - D(a_{n-1},\ldots,a_0)$ can have at most two roots in $E$, it's only roots are $\pm \delta(\alpha)$ so $\delta(\alpha)\in k$.
  \end{itemize}
\end{proof}
Finally we note a final fact about the discriminant.\begin{lemma}
  Let $n$ and $D$ be as before, let $k$ be any field and $f = X^n+a_{n-1}X^{n-1}+\cdots+a_0$ be any monic polynomial of degree $n$ over $k$. Then $f$ is inseperable (has at least one multiple root in $k^a$) if and only if $D(a_{n-1},\ldots,a_0) = 0$.
\end{lemma}
\begin{proof}
  Let $\alpha_1,\ldots,\alpha_n$ be the roots of $f$ then \begin{center}$\prod_{i<j}{(\alpha_i-\alpha_j)}^2 = D(a_{n-1},\ldots,a_0) = 0 \iff \alpha_i=\alpha_j$ for some $1\leq i<j\leq n \iff f$ has a multiple root in $k^a$\end{center}
\end{proof}
\end{bergman}
\paragraph*{Cubic extensions}
\begin{example}
Let $k$ be a field not of characteristic $2$ nor $3$. Let \[
  f(X) = X^3 + aX +b  
\]
Be any polynomial of degree $3$ can be brought into this form (depressed cubic). Assume that $f$ has no root in $k$. Then $f$ is irreducible. Let $\alpha$ be a root of $f(X)$, Then $[k(\alpha)\colon k] = 3$.

Let $K$ be the splitting field, since $\text{char}~k\neq 2,3$, $f$ is separable, so let $G$ be the Galois group.
Then $G$ has order $3$ or $6$ since $G$ is a subgroup of the symmetric group $S_3$. 
In the second case $k(\alpha)$ is not normal over $k$, this is because the corresponding group is the stablizer of $1\in \{1,2,3\}$ in $S_3$ which is not a normal subgroup.%%

\subparagraph*{How do we test whether the Galois group is the full symmmetric group?}
We will consider the discriminant, if $\alpha_1,\alpha_2,\alpha_3$ are the distinct roots of $f(X)$ we let\[
  \delta = (\alpha_1-\alpha_2)(\alpha_2-\alpha_3)(\alpha_1-\alpha_3) \text{ and }\Delta = \delta^2  
\]
If $G$ is the Galois groups and $\sigma\in G$, then $\sigma(\delta) = \pm \delta$. Hence $\sigma$ leaves $\Delta$ fixed. Thus $\Delta$ is in the ground field $k$. Furthermore we have seen that \[
  \Delta = -4a^3-27b^2  
\]
The set of $\sigma$ in $G$ which leave $\delta$ fixed is precisely the set of even permutations. Thus $G$ is the symmetric group if and only if $\Delta$ is not a square in $k$. We summarize this by saying:  
\begin{center}
    Let $f(X)$ be a cubic polynomial in $k[X]$, and assume  $char~k\neq 2,3$. Then \begin{enumerate}[label = (\alph*)]
    \item $f$ is irreducible over $k$ if and only if $f$ has no root in $k$.
    \item Assume $f$ irreducible. The the Galois group of  $f$ is $S_3$ if and only if the discriminant of $f$ ois not a square in $k$. If the discriminant is a square, then the Galois group is cuclic of order $3$, equal to the alternating group $A_3$ as a permutation of the roots of $f$.
\end{enumerate}
\end{center}
For instance, consider \[
    f(X) = X^3-X+1  
\]
over the rational numbers. Any rational root must be $1$ or $-1$, and so $f(X)$ is irreducible over $\Q$. The discriminant is $-23$ and is not a square. Hance the Galois group is the symmetric group. The splitting field contains a subfield of degree $2$, namely $k(\delta) = k(\sqrt{\delta})$.

On the hand, let $f(X) = X^3-3X+1$. Then  $f$ has no root in  $\Z$, whence no root in $\Q$, so $f$ is irreducible. The discriminant is $81$, a square, so the Galois group is cyclic of order $3$. 
\end{example}
\begin{example}[]
    We consider the polynomial $f(X) = X^4 - 2$ over the rationals  $\Q$. It is irreducible by Eisenstein's criterion. Let $\alpha$ be a real root. Let $i=\sqrt{-1}$. Then $\pm\alpha$ and $\pm i\alpha$ are the four roots of $f(X)$, and \[
    \Q(\alpha)\colon \Q = 4
\] 
Hence the splitting field of $f(X) $ is \[
    K=\Q(\alpha,i)
\]
The field $\Q(\alpha)\cap\Q(i)$ has degree $1$ or $2$ over $\Q$. 
The degree cannot be $2$ otherwise $i\in \Q(\alpha)$, which is impossible since $\alpha$ is real. Hence the degree is $1$. Hence $i$ has degree $2$ over $\Q(\alpha)$ and therefore $[K\colon \Q]=8$.
The Galois group of $f(X)$ has order $8$.

There exists an automorphism $\tau$ of $K$ leaving $\Q(\alpha)$ fixed, sending $i$ to $-i$, because $K$ is Galois over $\Q(\alpha)$, of degree 2.
Then $\tau^2 = id$.

\begin{equation*}
\begin{tikzcd}
   & \Q(\alpha,i) = K \ar[dr, dash, "4"]&\\
    \Q(\alpha) \arrow[ur, dash, "2"] &  & \Q(i) \ar[dl, dash, "2"]\\
    & \Q \ar[ul, dash, "4"]  &
\end{tikzcd}
\end{equation*}

\end{example}
